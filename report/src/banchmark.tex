\section{Тест производительности}
Существует также алгоритм сортировки подсчётом, работающий за квадратичное время. В нем не используется массив, содержащий информацию о элементах, меньших или равных данному. В данном алгоритме используется лишь один вспомогательный массив - результирующий.\\\\
Псевдокод алгоритма\cite{wikipedia_sort}\\
\begin{lstlisting}
SquareCountingSort
    for i = 0 to n - 1
        c = 0;
        for j = 0 to i - 1
            if A[j] <= A[i]
                c = c + 1;
        for j = i + 1 to n - 1
            if A[j] < A[i]
                c = c + 1;
        B[c] = A[i];
\end{lstlisting}

Напишем простую программу generate для генерации тестов(принимает количество тестовых строк, минимальный и максимальный ключи, которые гарантированно будут в тесте). Также добавим в программу код, выводящий в $std::cerr$ время выполнения в миллисекундах(используется библиотека std::chrono). Сравним данным способом время, за которое выполняют свою задачу $std::sort$, сортировка подсчётом и квадратичная сортировка подсчётом.

\begin{lstlisting} [language=C]
#include <ctime>
#include <random>
#include <limits>
#include <cstdlib>
#include <iostream>
#include <iomanip>
using namespace std;

default_random_engine rng;

uint64_t my_random(uint64_t max) {
    uniform_int_distribution<unsigned long long> dist_ab(0, max);
    return dist_ab(rng);
}

int main(int argc, char** argv) {
    if (argc < 4) {
        return 1;
    }
    long long count = stoll(argv[1]);
    long long min_key = stoll(argv[2]);
    long long max_key = stoll(argv[3]);
    cout << setw(6) << setfill('0') << min_key << my_random(numeric_limits<uint64_t>::max()) << "\n";
    cout << setw(6) << setfill('0') << max_key << "\t" << my_random(numeric_limits<uint64_t>::max()) << "\n";
    for (int i = 0; i < count; ++i) {
        cout << setw(6) << setfill('0') << my_random(999999 - min_key - max_key) + min_key << "\t" << 	          my_random(numeric_limits<uint64_t>::max()) << "\n";
    }
}
\end{lstlisting}


\subsection{Протокол тестирования производительности}
\begin{alltt}

ilya@ilya-lenovo:~/test_dir$ ./generate 1000 > test_1000
ilya@ilya-lenovo:~/test_dir$ ./generate 10000 > test_10000
ilya@ilya-lenovo:~/test_dir$ ./generate 100000 > test_100000
ilya@ilya-lenovo:~/test_dir$ ./generate 1000000 > test_1000000

ilya@ilya-lenovo:~/test_dir$ ./std_sort < test_1000 >result_1000
Std_sort_time: 1ms
ilya@ilya-lenovo:~/test_dir$ ./std_sort < test_10000 >result_10000
Std_sort_time: 5ms
ilya@ilya-lenovo:~/test_dir$ ./std_sort < test_100000 >result_100000
Std_sort_time: 59ms

ilya@ilya-lenovo:~/test_dir$ ./square_sort < test_1000 > result_1000
Square_counting_sort_time: 22ms
ilya@ilya-lenovo:~/test_dir$ ./square_sort < test_10000 > result_10000
Square_counting_sort_time: 1424ms
ilya@ilya-lenovo:~/test_dir$ ./square_sort < test_100000 > result_100000
Square_counting_sort_time: 125425ms

ilya@ilya-lenovo:~/test_dir$ ./da_01 < test_1000 > result_1000
Counting_sort_time: 21ms
ilya@ilya-lenovo:~/test_dir$ ./da_01 < test_10000 > result_10000
Counting_sort_time: 28ms
ilya@ilya-lenovo:~/test_dir$ ./da_01 < test_100000 > result_100000
Counting_sort_time: 49ms

ilya@ilya-lenovo:~/CLionProjects/da_01/cmake-build-debug$ ./da_01 < test_1000000 > result
Counting_sort_time: 366ms
ilya@ilya-lenovo:~/CLionProjects/da_01/cmake-build-debug$ ./std_sort < test_1000000 > result
Std_sort_time: 653ms


\end{alltt}

Как видно, время работы сортировки за квадратичное время очень быстро растёт с увеличением количества тестовых данных, в то время как сортировка подсчётом за линейное время работает крайне быстро. Как видно из последнего теста, время сортировки подсчётом для больших наборов тестовых данных становится значительно быстрее стандартной сортировки(ожидаемо, $std::sort$ имеет оценку $O(nlogn)$.

\pagebreak

